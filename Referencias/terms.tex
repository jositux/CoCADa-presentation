\newglossaryentry{CAD}
{
    name=CAD,
    description={Computer Aided Design, Diseño Asisitido por Computadora}
}

\newglossaryentry{CAM}
{
    name=CAM,
    description={Computer Aided Manufacturing, Fabricación Asistida por Computadora}
}

\newglossaryentry{CNC}
{
    name=CNC,
    description={Control Numérico Computarizado}
}

\newglossaryentry{FLOSS}
{
    name=FLOSS,
    description={Free Libre Open Source Software}
}

\newglossaryentry{CPD}
{
    name=CPD,
    description={Collaborative  Product  Development, Desarrollo Colaborativo de Productos}
}

%\newglossaryentry{Co-Diseno}
%{
%   name=Co-Diseño,
%   description={Práctica que permite al usuario final %participar en todo el proceso de desarrollo de un %proyecto de diseño}
%}

\newglossaryentry{WebGL}
{
    name=WebGL,
    description={Web Graphics Library, Librería de Gráficos en la Web}
}

\newglossaryentry{UX}
{
    name=UX,
    description={User Experience, Experiencia de Usuario}
}

\newglossaryentry{Lean UX}
{
    name=Lean UX,
    description={Metodología de trabajo enfocada en la  experiencia de usuario (UX)}
}

\newglossaryentry{GUI}
{
    name=GUI,
    description={Graphical User Interface, Interfaz Gráfica de Usuario}
}

\newglossaryentry{OpenGL}
{
    name=OpenGL,
    description={Open Graphics Library, Librería de Gráficos Abierta.}
}

\newglossaryentry{CSG}
{
    name=CSG,
    description={Constructive Solid Geometry, Geometría Constructiva de Sólidos. Modelo de representación de objetos 3D en forma de árbol}
}

\newglossaryentry{API}
{
    name=API,
    description={Application Programming Interface, Interfaz de Programación de Aplicación}
}

\newglossaryentry{Blender}
{
    name=Blender,
    description={Software de código abierto para creación de gráficos 3D}
}

\newglossaryentry{Rhino}
{
    name=Rhino,
    description={Herramienta de software para modelado en 3D basado en NURBS}
}


\newglossaryentry{Grasshopper}
{
    name=Grasshopper,
    description={Lenguaje de programación visual para modelado integrado en Rhino}
}

\newglossaryentry{OpenSCAD}
{
    name=OpenSCAD,
    description={Software de modelado 3D basado en un lenguaje de descripción textual}
}

\newglossaryentry{OpenJSCAD}
{
    name=OpenJSCAD,
    description={Software de modelado 3D en la web basado en el lenguaje javascript}
}

%%%%%%%%%%%%
\newglossaryentry{lightgl.js}
{
    name=lightgl.js,
    description={Librería para desarrollar aplicaciones WebGL}
}

\newglossaryentry{CSG.js}
{
    name=CSG.js,
    description={Librería para realizar operaciones de modelado CSG en la web}
}

\newglossaryentry{AutoCAD}
{
    name=AutoCAD,
    description={Software CAD utilizado para dibujo 2D y modelado 3D}
}

\newglossaryentry{AutoLisp}
{
    name=AutoLisp,
    description={Lenguaje de programación derivado del lenguaje Lisp para rutinas de AutoCAD}
}

\newglossaryentry{SolidWorks}
{
    name=SolidWorks,
    description={Software CAD para modelado mecánico en 2D y 3D}
}

\newglossaryentry{OnShape}
{
    name=OnShape,
    description={Plataforma de diseño de productos que combina herramientas CAD, PDM y colaboración en la nube}
}

\newglossaryentry{FeatureScript}
{
    name=FeatureScript,
    description={Lenguaje de programación diseñado para interactuar con OnShape}
}

\newglossaryentry{Modelo.io}
{
    name=Modelo.io,
    description={Plataforma web para presentación y colaboración orientada a arquitectos e ingenieros que trabajan con CAD}
}

\newglossaryentry{Speckle}
{
    name=Speckle,
    description={Proyecto que permite a los usuarios de programas como Grasshopper, compartir diseños en la web}
}

\newglossaryentry{PMV}
{
    name=PMV,
    description={Producto Mínimo Viable. Producto con suficientes características para satisfacer a los usuarios y proporcionar feedback para desarrollos futuros}
}

\newglossaryentry{feedback}
{
    name=feedback,
    description={Palabra del inglés que significa retroalimentación; se utiliza como sinónimo de respuesta o reacción}
}

\newglossaryentry{zoom}
{
    name=zoom,
    description={Capacidad de aumentar o disminuir la escala en una figura o texto}
}

\newglossaryentry{ASCII}
{
    name=ASCII,
    description={American Standard Code for Information Interchange, Código Estándar Estadounidense para el Intercambio de Información}
}


\newglossaryentry{PDM}
{
    name=PDM,
    description={Product Data Management. Gestión de Datos del Producto}
}

\newglossaryentry{WPDM}
{
    name=WPDM,
    description={Web Product Data Management. Gestión de Datos del Producto basado en la web}
}


\newglossaryentry{STEP}
{
    name=STEP,
    description={Standard for the Exchange of Product Data. Formato de intercambio de datos utilizado para representar objetos 3D CAD e información relacionada}
}

\newglossaryentry{STL}
{
    name=STL,
    description={STereoLithography. Formato de archivo que define geometría de objetos 3D, excluyendo información como color, texturas o propiedades físicas}
}

\newglossaryentry{CGCAL}
{
    name=CGCAL,
    description={Computational Geometry Algorithms Library. Librería de C++ que provee una colección de estructuras de datos  y algoritmos para geometría}
}

\newglossaryentry{Open-Cascade}
{
    name=Open-Cascade,
    description={Open Computer Aided Software for Computer Aided Design and Engineering}
}

\newglossaryentry{python}
{
    name=python,
    description={Lenguaje de programación}
}

\newglossaryentry{PHP}
{
    name=PHP,
    description={Hypertext Preprocessor, preprocesador de hipertexto. Lenguaje de programación}
}

\newglossaryentry{DE}
{
    name=DE,
    description={Data Exchange, Intercambio de Datos}
}

\newglossaryentry{FB}
{
    name=FB,
    description={Featured Based, Basado en Características}
}

\newglossaryentry{Web Services}
{
    name=Web Services,
    description={Servicios Web. Tecnologías que utiliza un conjunto de protocolos y estándares para intercambiar datos entre aplicaciones}
}

\newglossaryentry{FBDE}
{
    name=FBDE,
    description={Featured Based Data Exchange, Intercambio de Datos Basado en Características}
}


\newglossaryentry{REST}
{
    name=REST,
    description={REpresentational State Transfer, Transferencia de Estado Representacional}
}

\newglossaryentry{SOAP}
{
    name=SOAP,
    description={Simple Object Access Protocol. Formato de mensaje XML utilizado en interacciones de servicios web}
}

\newglossaryentry{Framework}
{
    name=framework,
    description={Marco de trabajo. Conjunto estandarizado de conceptos, prácticas y criterios para enfocar un tipo de problemática particular que sirve como referencia, para resolver problemas similares}
}

\newglossaryentry{Node.js}
{
    name=Node.js,
    description={Entorno JavaScript de lado de servidor que utiliza un modelo asíncrono y dirigido por eventos}
}

\newglossaryentry{Express.js}
{
    name=Express.js,
    description={Framework de desarrollo de aplicaciones web para Node.js}
}

\newglossaryentry{LoopBack}
{
    name=Loopback,
    description={Framework basado en Express.js orientado a construir REST APIs}
}

\newglossaryentry{NoSQL}
{
    name=NoSQL,
    description={Sistemas de gestión de bases de datos que difiere del modelo SGBDR (Sistema de Gestión de Bases de Datos Relacionales)}
}

\newglossaryentry{JavaScript}
{
    name=JavaScript,
    description={Lenguaje de programación}
}


\newglossaryentry{JSON}
{
    name=JSON,
    description={JavaScript Object Notation, Notación de objeto de JavaScript. Formato de texto sencillo para el intercambio de datos}
}

\newglossaryentry{Vue.js}
{
    name=Vue.js,
    description={Framework JavaScript progresivo para crear interfaces de usuario y aplicaciones web}
}

\newglossaryentry{Nuxt.js}
{
    name=Nuxt.js,
    description={Framework para aplicaciones universales basado en  Vue.js}
}

\newglossaryentry{Material Design}
{
    name=Material Design,
    description={Sistema de Diseño Modular para dispositivos digitales que toma criterios de comportamiento y forma basados en las leyes de la realidad, entre ellos la iluminación, movimiento y materiales}
}

\newglossaryentry{Vuetify}
{
    name=vuetify,
    description={Framework basado en Vue.js que implementa las normativas de Material Design}
}

\newglossaryentry{MongoDB}
{
    name=MongoDB,
    description={Sistema de base de datos NoSQL orientado a documentos}
}

\newglossaryentry{GitHub}
{
    name=GitHub,
    description={Plataforma web para alojar proyectos utilizando el sistema de control de versiones Git}
}


\newglossaryentry{EndPoints}
{
    name=EndPoints,
    description={Puntos de conexión de un servicio, herramienta o aplicación al que se accede a través de una API}
}

\newglossaryentry{URL}
{
    name=URL,
    description={Uniform Resource Locator, Localizador de Recursos Uniforme}
}




\newglossaryentry{HTTP}
{
    name=HTTP,
    description={Hypertext Transfer Protocol, Protocolo de Transferencia de Hipertexto}
}

\newglossaryentry{Ace Editor}
{
    name=Ace Editor,
    description={Editor de código basado en la web, escrito en JavaScript}
}

\newglossaryentry{Three.js}
{
    name=Three.js,
    description={Librería escrita en JavaScript para crear gráficos y animaciones 3D en un navegador web}
}

\newglossaryentry{Shared Workspace}
{
    name=Shared Workspace,
    description={Marco de Trabajo Compartido}
}

%\newglossaryentry{Design Thinking}
%{
%    name=Design Thinking,
%    description={Pensamiento de diseño. Hace referencia a %los procesos cognitivos, estratégicos y prácticos %mediante los cuales se elaboran los conceptos %relacionados con el diseño}
%}

\newglossaryentry{Agile}
{
    name=Agile,
    description={Enfoque para la toma de decisiones en los proyectos de software, que se refiere a métodos basados en el desarrollo iterativo e incremental}
}

\newglossaryentry{SCRUM}
{
    name=SCRUM,
    description={Framework para el desarrollo ágil de software}
}

\newglossaryentry{Sprint}
{
    name=Sprint,
    description={Intervalo prefijado en SCRUM durante el cual se crea un incremento de producto ``Hecho o Terminado" utilizable, potencialmente entregable}
}

%\newglossaryentry{Iterative Design}
%{
%    name=Iterative Design,
%    description={Diseño Iterativo. Refinamiento constante del diseño basado en las pruebas con el usuario y otros métodos de evaluación}
%}

%\newglossaryentry{Lean Startup}
%{
%    name=Lean Startup,
%    description={Metodología que apunta a acortar los ciclos de desarrollo de productos adoptando una combinación de experimentación impulsada por hipótesis para medir el progreso}
%}


\newglossaryentry{Web Components}
{
    name=Web Component,
    description={Conjunto de características para la creación de widgets o componentes reutilizables en documentos y aplicaciones web}
}

\newglossaryentry{file storage}
{
    name=file storage,
    description={Almacenamiento de archivos}
}

\newglossaryentry{OAuth}
{
    name=OAuth,
    description={Estándar abierto que permite flujos simples de autorización para aplicaciones informáticas}
}


\newglossaryentry{APP}
{
    name=APP,
    description={Aplicación informática}
}


\newglossaryentry{Back-End}
{
    name=Back-End,
    description={Parte de la aplicación que corre del lado del servidor}
    %description={Es la parte que procesa la entrada desde el front-end en el servidor}
}

\newglossaryentry{Front-End}
{
    name=Front-End,
    description={Parte de la aplicación web que interactúa con los usuarios }
}

\newglossaryentry{DBMS}
{
    name=DBMS,
    description={DataBase Management System, Sistema de Gestión de Base de Datos}
}

\newglossaryentry{HTML}
{
    name=HTML,
    description={HyperText Markup Language, Lenguaje de Marcado de Hipertexto}
}

\newglossaryentry{HTML5}
{
    name=HTML5,
    description={HTML en su Versión 5}
}

\newglossaryentry{CSS}
{
    name=CSS,
    description={Cascading Style Sheet, Hoja de Estilos en Cascada}
}

\newglossaryentry{SSR}
{
    name=SSR,
    description={Server Side Rendering, Renderizado en el Lado del Servidor}
}

\newglossaryentry{Sketch}
{
    name=sketch,
    description={Boceto o bosquejo}
}

\newglossaryentry{research}
{
    name=research,
    description={Investigación}
}



\newglossaryentry{slider}
{
    name=slider,
    description={Control deslizante. Permite a los usuarios realizar selecciones a partir de un rango de valores}
}

\newglossaryentry{login}
{
    name=Login,
    description={Ingresar. Proceso mediante el cual se controla el acceso a un sistema}
}

\newglossaryentry{API Explorer}
{
    name=API Explorer,
    description={Explorador de API}
}

\newglossaryentry{scripting}
{
    name=scripting,
    description={Interfaces de secuencias de comandos}
}