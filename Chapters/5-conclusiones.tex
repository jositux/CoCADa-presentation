\cleardoublepage
%-----------------------------------------------------
% Chapter 5: Conclusiones
%-----------------------------------------------------
\chapter{Conclusiones}
\label{chap: cap5}

%Según lo investigado, la tendencia actual en el desarrollo de productos es la colaboración (co-diseño) entre personas con diferentes competencias y dispersas geográficamente. Por esto, surgió la necesidad de desarrollar una solución distribuida en la que los participantes puedan colaborar independientemente de su profesión.\vskip
%Las herramientas CAD analizadas, en algunos casos no soportan la colaboración o son aplicaciones orientados a un perfil específico de profesionales (ingenieros, arquitectos, diseñadores industriales, etc).\vskip

El uso de \textit{Lean UX} %como metodología orientada a la experiencia de usuario (UX) permitió una comprensión de las necesidades reales.
%La utilización de la misma 
durante el desarrollo %y las iteraciones o experimentos con  prototipos de alta fidelidad (PMV) 
ha permitido evaluar la solución de forma objetiva y obtener \textit{feedback} de inmediato con dos usuarios de diferentes competencias. Como resultado se obtuvo una aplicación con una experiencia de usuario (UX) satisfactoria.\vskip
%Los experimentos con dos usuarios de diferentes competencias %utilizando prototipos de alta fidelidad (PMV) 
%analizados en la sección \ref{pmv-cocada}) han permitido validar las hipótesis de la solución. 
Los experimentos Demo \#1 y Demo \#1.1 (ver sección \ref{pmv-cocada}) demostraron que el sistema es satisfactorio para el usuario Persona A (sin conocimientos en CAD) en las tareas de visualizar modelos 3D y manipular sus parámetros.
Por otro lado, mediante Demo \#2 y Demo \#2.1 (ver sección \ref{pmv-cocada}) se verifica que el usuario Persona B (profesional) puede crear modelos mediante \textit{scripting}, definir parámetros y diseñar nuevas primitivas.
%, facilitando así el auto-aprendizaje sin necesidad de un entrenamiento previo. 
%Los experimentos analizados en la sección \ref{pmv-cocada} demostraron que el sistema es satisfactorio para el usuario Persona A (sin conocimientos en CAD) en las tareas de visualizar modelos 3D y manipular sus parámetros.
%como el modelado mecánico 
%y también para uso recreativo.
%, por ejemplo en el diseño de piezas simples para impresión 3D.
\vskip
%Los usuarios con experiencia en CAD (diseñadores) necesitan crear modelos 3D en un entorno que permita automatizar partes del diseño, de manera que se pueda registrar su evolución. 
%En el capítulo \ref{chap: cap2} para tener una mejor comprensión de las herramientas necesarias para dicho entorno, se exploraron algunas aplicaciones de diseño paramétrico especificado en algoritmos, incluido OpenJSCAD. Además, deben poder comunicarse de forma eficiente con otros usuarios (no expertos). \vskip
%Por otro lado, los usuarios no expertos necesitan de un espacio de trabajo para expresar sus ideas y experimentar de forma intuitiva  con  modelos 3D. 

%; a través de la manipulación directa, comentarios, anotaciones y  obtención de archivos (STL) preparados para la fabricación digital.

%En este sentido, en el capítulo \ref{chap:cap3} se analizaron algunas aplicaciones de referencia que 

%que
%implementan estas funcionalidades. %

%permiten explorar modelos 3D, realizar comentarios, anotaciones, exportar archivos, etc. 
En CoCADa se implementaron características de \textit{WPDM} (gestión de los datos del producto basados en la web) para que sea un sistema CAD distribuido.
%, brindando acceso a usuarios dispersos geográficamente, independientemente de las plataformas de software que utilicen. 
El registro de la evolución de los diseños se ha logrado mediante la incorporación de los conceptos diseño paramétrico, \textit{scripting} y \textit{FBDE} (intercambio de datos basado en características). Estos se integraron al \textit{WPDM} como versiones del producto (también llamado ``árbol de historia''). A su vez, en cada versión (espacio de trabajo compartido) permite la colaboración mediante las herramientas para manipular los diseños, comentar, adjuntar archivos y realizar anotaciones sobre los modelos.\vskip

El uso de herramientas FLOSS ha sido fundamental para adaptar el código fuente de otras aplicaciones a las necesidades del proyecto.
Las tecnologías (Node.js, Nuxt.js, LoopBack, Vue.js, Vuetify) permitieron emplear el mismo lenguaje de programación JavaScript en toda %s las partes de 
la aplicación (\textit{Back-End} y \textit{Front-End}), evitando la complejidad de mantener código de diferentes lenguajes.\vskip

\clearpage

Entre los trabajo futuros se considera:

\begin{itemize}
\item Implementar comentarios y características propias de redes sociales como \textit{@nombre} para hacer referencia a los usuarios destinatarios del mensaje, notificaciones, mensajes de audio, etc.\vskip

\item  Extender los modelos sólidos a otras representaciones mediante librerías como \Gls{Three.js}\footnote{\url{https://threejs.org/}} y así incorporar otras características a la visualización de productos como ser texturas, iluminación, transparencias, etc.

\item Implementar un editor basado en nodos para programar modelos 3D de forma similar a Grasshopper.

\item Brindar soporte a espacios de trabajo compartido en tiempo real (Al mismo Tiempo, en Sitios Diferentes) visto en la sección \ref{section:colabo}. De esta manera, los cambios en un modelo se reflejarían automáticamente para todos los participantes.

\item Investigar y desarrollar una aplicación web colaborativa que implemente herramientas de esculpido digital\footnote{El esculpido digital es una disciplina que conjuga la técnica tradicional de modelado con materiales blandos (arcilla, por ejemplo) con las nuevas tecnologías.}.


\end{itemize}