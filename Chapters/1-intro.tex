%-----------------------------------------------------
% Chapter 1: Introducción
%-----------------------------------------------------
\cleardoublepage
\chapter{Introducción}
\label{chap:cap1}
Los sistemas de \textbf{diseño asistido por computadora} en inglés \textit{Computer Aided Design} (\Gls{CAD}) tienen una amplia trayectoria y han demostrado ser excelentes herramientas para el diseño de productos \citep{Chao2001}, como característica distintiva incorporan el paradigma de Diseño Paramétrico \citep{Davis2013} que posibilita la modificación del diseño de manera sencilla y rápida mediante el ajuste de sus parámetros; sin necesidad de modelar todo nuevamente.
En la industria de la manufactura estos sistemas son indispensables, ya que combinados con la \textbf{Fabricación Asistida por Computadora} en inglés \textit{Computer Aided Manufacturing}  (\Gls{CAM}) y el \textbf{Control Numérico Computarizado} (\Gls{CNC}), permiten que la fabricación digital \citep{Chryssolouris2009} se pueda aplicar a prácticamente cualquier producto. La innovación en este contexto es la vinculación directa entre el modelo CAD y su fabricación CAM. 

En Argentina, tal es la interés en este campo, que desde el año 2013, el ministerio de ciencia, tecnología e innovación productiva ha impulsado diversas acciones para la difusión, capacitación y apoyo a proyectos de innovación, desarrollo y adopción de estas tecnologías. Sobre todo en el área de la impresión 3D \citep{BERMAN2012155}, al ser una de las más difundidas y promisorias \citep{MinisteriodeCiencia2015}.


La masificación de la fabricación digital hace necesario el \textbf{desarrollo colaborativo de productos} en inglés \textit{Collaborative Product Development} (\gls{CPD}) \citep{Elfving2007} entre grupos de expertos con diferentes competencias y muchas veces geográficamente dispersos. Este enfoque, también conocido como \textbf{co-diseño} \citep{PerezGarcia2014} requiere de una comunicación efectiva en entornos distribuidos\footnote{Entorno en el que los sistemas informáticos en red colaboran aportando sus recursos.}; dado que los problemas en los procesos de diseño suelen ser ocasionados por errores en la interpretación de ideas entre los participantes.

%Por otra parte, \textquote{\textit{la mayoría de los sistemas CAD/CAM han sido diseñados para trabajar en entornos aislados, no permitiendo comunicación alguna con otros sistemas}}\citep{Chao2001}. 

Por otra parte, la mayoría de los sistemas CAD/CAM son aplicaciones de escritorio  %para trabajar en entornos aislados, %no permitiendo comunicación alguna con otros sistemas 
%Este hecho se debe a la naturaleza propia de la aplicaciones 
que necesitan ser instaladas obligatoriamente en cada computadora por separado. Las aplicaciones web son menos propensas a este %problema 
hecho ya que el usuario accede a ellas sin necesidad de instalar el software en su dispositivo. El uso del CAD en la web se ha incrementado en los últimos años gracias a la estandarización de tecnologías como \Gls{HTML5}\footnote{\url{https://www.w3.org/TR/html52/}} y \Gls{WebGL} \citep{Nyman2013} que permiten experiencias de usuario en inglés \textit{User Experience} (\Gls{UX}) \citep{hartson2012ux} similares a las de las aplicaciones de escritorio. \textquote{\textit{Uno de los avances más recientes en este campo es la capacidad de las aplicaciones web para representar gráficos en 3D.}} \citep{Waerner2012}. \vskip


Recientemente han aparecido servicios que proporcionan funcionalidades de CAD y administración de proyectos de diseño en la nube \citep{mell2011nist}. Un ejemplo es \Gls{OnShape}\footnote{\url{https://www.onshape.com/}} que permite a los equipos colaborar mediante modelos compartidos desde la web. Sin embargo, esta plataforma está orientada a especialistas en el diseño de piezas mecánicas, siendo ideal para los ambientes de ingeniería y diseño industrial pero difícil de utilizar por personas sin formación en el área de modelado, en consecuencia, dificulta el co-diseño en términos de diversidad de competencias.\vskip
Esta situación provoca que los participantes opten por utilizar servicios colaborativos genéricos para comunicarse, como la mensajería instantánea, las redes sociales o videoconferencias mediante Skype\footnote{\url{https://www.skype.com/}} y compartan sus archivos utilizando Dropbox\footnote{\url{https://www.dropbox.com/}} o sistemas similares. La multiplicidad de medios produce sesgo en la comprensión de los proyectos a nivel general.
\textquote{\textit{Para un equipo es preferible contar con la colaboración integrada en una misma aplicación web}} \citep{Alfaiate2017}. Otras limitaciones de estos servicios son la falta de mecanismos para registrar el progreso o cambios de los diseños en todo momento y la imposibilidad de señalar o ``marcar'' los problemas detectados de forma precisa, de manera que se pueda informar sobre estos al resto del equipo.\vskip


%\clearpage
Ante los problemas planteados, una \textbf {aplicación web colaborativa y distribuida para el diseño de productos orientados a la fabricación digital} proporcionaría herramientas para facilitar el co-diseño entre participantes con diferentes competencias y mejorar la administración en términos del progreso o evolución de los diseños. La aplicación o prototipo de software lleva el nombre de \textbf{Aplicación de CAD Colaborativa} en inglés \textit{Collaborative CAD Application} (CoCADa). \vskip


\vspace{0mm}

Este documento se encuentra organizado de las siguiente manera: 
En el \textbf{Capítulo 2} se explican las características de los sistemas CAD y los conceptos fundamentales de los  modelos orientados a la fabricación digital. Luego se analizan las ventajas del uso de \textit{\Gls{Lean UX}} \citep{Gothelf2013} para desarrollar un software colaborativo y distribuido (\textbf{Capítulo 3}), se enuncia el problema y los detalles de la solución (CoCADa) explicando los componentes constitutivos y las principales funcionalidades implementadas (\textbf{Capítulo 4}). Finalmente, en el \textbf{Capítulo 5}, se realizan la conclusiones y se formulan directivas para trabajos futuros.









%\clearpage

\section{Objetivos}

\subsection{Objetivo general}

Desarrollar un prototipo de sistema que facilite la colaboración en el proceso de diseño de productos entre personas con diferentes competencias.


\subsection {Objetivos Específicos}
\begin{itemize}
  \item Recabar bibliografía sobre sistemas CAD paramétricos, modelos 3D en la web, modelos orientados a la fabricación digital, sistemas colaborativos y distribuidos, metodología \textit{Lean UX}. 
  
  \item Analizar y describir soluciones de CAD existentes, sobre todo aquellas que se ofrecen como servicios en la nube.
  
  \item Indagar sobre tecnologías colaborativas para el diseño iterativo de modelos 3D.
  
  \item Investigar los requerimientos del sistema y los componentes de software a utilizar en el desarrollo.
  
  \item Diseñar la interfaz gráfica de usuario en inglés \textit{Graphical User Interface} (\Gls{GUI}) de la aplicación utilizando el enfoque \textit{Lean UX}.
  
  \item Diseñar un modelo de dominio de una capa de servicios.
  
  \item Desarrollar el prototipo de software con herramientas \Gls{FLOSS} \citep{stallman2007software}\citep{Stallman} según los requerimientos de la GUI y el Modelo de dominio.
  
  \item Realizar pruebas para experimentar en diferentes escenarios y evaluar los resultados.
\end{itemize}
