% ADDITIONAL DECLARATIONS HERE (IF ANY)

%\vskip5mm
%Signature:
%\vskip20mm
%AUTOR
%José María Guaimas



%-----------------------------------------------------
% Resúmen 2
%-----------------------------------------------------
\addcontentsline{toc}{section}{Resumen}
\begin{center}
\section*{Resumen}
\end{center}
\vskip

%Los algoritmos de aprendizaje automático son muy útiles para la clasificación de datos de series de tiempo en astronomía en esta era de publicación de datos de encuestas públicas a escala peta. Estos métodos pueden facilitar el descubrimiento de nuevos eventos desconocidos en la mayoría de las áreas astrofísicas, así como mejorar el análisis de muestras de fenómenos conocidos. Los algoritmos de aprendizaje automático utilizan características extraídas de los datos recopilados como variables predictivas de entrada. Una herramienta pública llamada Análisis de características para series de tiempo (FATS) ha demostrado ser una excelente herramienta para la extracción de características, particularmente la clasificación de curvas de luz para objetos variables. En este estudio, presentamos una mejora importante para FATS, que corrige las opciones de diseño inconvenientes, detalles menores y documentación para el proceso de reingeniería. Esta mejora comprende un nuevo paquete de Python llamado feets, que es importante para futuras refactorizaciones de código para herramientas de software astronómicas.
%Palabras clave: astroinformática, algoritmo de aprendizaje automático: selección de características, software y su ingeniería: software
%problema posterior al desarrollo

%En el pasado el proceso de fabricar un producto consistía en delegar el diseño o enviar planos a un especialista para que realice la manufactura, los contactos eran presenciales y mantener el proyecto bajo control obligaba a realizar viajes frecuentes. 

%En la actualidad el contexto es muy diferente: 
El Desarrollo Colaborativo de Productos y el Co-Diseño son conceptos muy utilizados en las organizaciones, en especial, en las áreas que involucran el diseño y fabricación asistido por computadora \textit{CAD/CAM}. La evolución de las tecnologías web ha permitido la colaboración entre personas dispersas geográficamente, de diferentes campos de especialización e incluso sin formación en diseño. \vskip
Al mismo tiempo, la diversidad de conocimientos trae como consecuencia algunos problemas en la gestión de los proyectos, entre ellos: la comunicación imprecisa, la múltiple interpretación de ideas y la complejidad para registrar el progreso o cambios en los diseños. 

El presente trabajo propone el desarrollo de un prototipo de aplicación web que provee un marco para la colaboración multidisciplinaria mediante revisiones de modelos 3D. Estableciendo así, un tipo de comunicación entre los usuarios que va más allá de la geometría. 

El software llamado \textit{Colaborative CAD Application} (CoCADa) se desarrolla en base al enfoque \textit{Lean UX} y utiliza tecnologías \textit{Free Libre Open Source Software} (FLOSS). \vskip
 \vskip
%El software COCADA se encuentra disponible para la descarga en \url{https://github.com/jositux/Tesis_Josi_Marcos}  bajo licencia GNU General Public License (GPL)\footnote{\url{https://www.gnu.org/licenses/gpl-3.0.en.html}}. 

\vskip5mm
\textbf{Palabras Claves}: Co-diseño, CAD/CAM, Diseño Paramétrico, Modelado Sólido, WPDM, FBDE, Javascript, Lean UX.




	
% ADDITIONAL DECLARATIONS HERE (IF ANY)

%\vskip5mm
%Signature:
%\vskip20mm
%AUTOR
%José María Guaimas




%\textbf{Conceptos de Loopback}
%https://yo.toledano.org/desarrollo/loopback-conceptos-basicos/